\documentclass[a4paper, 12pt]{scrartcl}

\usepackage[left=2.5cm,top=3cm, bottom=3cm, right=2cm]{geometry}

\usepackage{secdot} % Dots in Section Numbers
\usepackage[utf8]{inputenc}
\usepackage[T1]{fontenc}
\usepackage[ngerman]{babel}
\usepackage[european]{circuitikz}
\usepackage{graphicx} % to include Pictures
\usepackage{float}

\usepackage{fancyhdr}

\usepackage{todonotes}
\usepackage{hyperref}

\pagestyle{fancy}
\fancyhf{}
\lhead{kawa-i}
\chead{Bebarengan Sepur: Entlang der Schienen}
\rhead{\thepage}
%\cfoot{\thepage}


\definecolor{nicegreen}{rgb}{0.09, 0.45, 0.27}

\begin{document}
    \begin{titlepage}
        \begin{center}
            \large{Bebarengan Sepur}
        \end{center}
    \end{titlepage}
    \tableofcontents

    \newpage

    \section{Einleitung}
    \textit{Bebarengan Sepur - Entlang der Schienen} ist ein philosophisches Bildungsliverollenspiel, welches versucht, die Themen Freiheit, Autonomie und Befreiung in einen Kontext zu bringen mit den Problemen des 21. Jahrhunderts (Klimakatastrophen, Wirtschaftskrisen, Rechtsradikalismus etc.). 
    Wir verbinden charakterbasiertes Liverollenspiel, angelehnt an \hyperref[nordic-larp]{Nordic Larp}, mit Kunst, Abstraktionen und surrealen Elementen. 
    Die Erzählstruktur des Liverollenspiels greift Element genealogischer Kritik auf. 
    Damit bilden \hyperref[kunst-abstraktion]{Kunst und Abstraktion}, sowie \hyperref[genealogische-kritik]{genealogische Kritik} die beiden Hauptmethoden zur Darstellung der Bildungsinhalte. 
    Diese zwei Methoden haben wir durch die \hyperref[java-reihe]{Java-Reihe} hinweg entwickelt und weiterentwickelt und sollen in \textit{Bebarangan Sepur} das erste Mal zur vollen und spieltragenden Entfaltung kommen. 
    So dienen sie hier nicht ausschließlich zu Darstellung der Bildungsinhalte, sondern bilden einen Hauptbestandteil bei der Kontruktion des Plots und der Spielgeschichte, sowie bei Erschaffung der Spielwelt.\\
    Die Themen Freiheit, Autonomie und Befreiung greifen wir anhand der Perspektiven neuerer und älterer Kritischer Theorie (Adorno, Menke, Saar), sowie ihrer Vordenker*innen (Hegel, Marx, Nietzsche) auf. 
    Theoretischer Hintergrund ist die Notwendigkeit der Befreiung aus der zweiten Natur, zum Einen als einzigem wirklichen Moment der Freiheit (Freiheit als Negation konkreter Unfreiheit), zum Anderen um fundierte Kritik an aktuellen Problemen zu formulieren.

    \section{Game-Design}
    \subsection{Spielsetting}
    Das Spiel spielt in einer (dystopischen) Zukunft in der Menschen, mittels eines Ausweises \"ahnlich dem heutigen Organspendeausweis, sich anstelle eines verordneten Gefängnisaufenthalts bereit erkl\"aren an Experimenten (oder auch \glqq Feldstudien\grqq) teilzunehmen. 
    Das Spiel besteht aus zwei einzelnen \glqq Akten\grqq, die jeweils zwei verschiedene solcher Experimente darstellen.
    \subsection{Die Experimente}
    Wieso gibt es die Experimente? 
    Wie genau sehen sie aus? 
    Was genau wird bei einem Experiment erforscht? 
    Von wem wird es durchgeführt? \\
    \subsection{Die Kollektive}
    Im Laufe des Spiels bilden die Spieler*innen anhand ihrer Charakterbeschreibungen, also vorherigen Zuordnung, mehrere (vermutlich drei) Kollektive. 
    Innerhalb dieser Kollektive bildet sich ein jeweiliger Kollektiv-Konsens (wie genau das passiert, ist noch unklar), man könnte auch davon sprechen, dass sich in jedem Kollektiv eine \hyperref[zweite-natur]{\glqq zweite Natur\grqq{}} bildet. 
    Das Eingehen jedes einzelnen Charakters in das Kollektiv, soll sich auch im Charakter widerspiegeln. Beispielsweise in seinem Raum. Die Identität des Kollektivs speist sich aus der der Individuen, gleichzeitig verändert sie aber auch die Individuen selbst. 
    Hier wird versucht ein Aspekt Hegels-Philosophie zu verdeutlichen. Bezug genommen wird speziell auf zwei Passagen der Hegelschen Rechtsphilosophie: \\
    \glqq Das objektiv Sittliche das an die Stelle des Abstrakten Guten tritt, ist durch die Subjektivität [...] konkrete Substanz.\grqq{} (S.161)\cite{BuchHegel} \\
    Und:\\
    \glqq Andererseits sind sie [die Gesetze des des objektiv Sittlichen] dem Subjekte nicht ein Fremdes, sondern es gibt das Zeugnis des Geistes von ihnen als von seinem eigenen Wesen.\grqq{} (S. 162)\cite{BuchHegel} \todo{Welches Buch? Einf\"ugen als echtes Zitat}
    
    \subsection{Ein Liverollenspiel in vier Akten}
    Das Liverollenspiel durchläuft vier Akte die jeweils mehrere Ebenen beinhalten: 1) das Spielsetting
    \subsubsection{Akt 1}
    Die Spieler*innen befinden sich alleine in ihrem Zugabteil, welches zugleich 
    \subsubsection{Akt 2}
    \subsubsection{Akt 3}
    \subsubsection{Akt 4}
   
    \subsection{Die Öffnung/ Der Cut}
    Erreichen die Spieler*inner innerhalb des zweiten Aktes des ersten Teils einen bestimmten Punkt, öffnen sich erstmals die beiden Abteile/ Etagen. 
    Es beginnt ein Countdown in Sekunden, begleitet von \textit{koyaanisqatsi} von Philipp Glas. 
    Während dieser Phase können sich erstmals alle Charaktere begegnen. 
    Am Ende des Countdowns wird ein \hyperref[cut]{Cut} initiert, das Spiel geht kurz OT und alle Spieler*innen werden wieder auf ihr Zimmer geschickt. 
    Alle Spieler*innen gehen schlafen.  
    Am nächsten Morgen, oder noch vor dem Schlafen erhalten die Teilnehmer*innen über Audio folgende Informationen:
    \begin{itemize}
    \item Hingergrundgeschichte, was ist passiert ist (Verschiedene Beschreibungen, z.B. Zugunglück, nun bist du im Krankenhaus, das war nur ein Traum, du bist jetzt Krankenschwester, du hast dich vor Jahren in das Krankenhaus als Patient eingeschlossen)
    \item Neue Beziehungen und damit auch so etwas wie neue Ziele
    \end{itemize}
    Übernacht wird das Setting zum Krankenhaus umgebaut. 
    Das Spiel beginnt wieder in Akt II.


    \section{Part One - Der Zug}
    \subsection{Spielsetting}
    \subsection{Spielrealität}
    \subsection{Metaebene}
    \subsection{Mechanismen}

    \section{Part Two - Die Anstalt}
    \todo{evtl anderer Name für \glqq die Anstalt\grqq{}}
    \subsection{Spielsetting}
    \subsection{Spielrealität}
    \subsection{Metaebene}
    \subsection{Mechanismen}

    \section{Textadventure}
    \todo{Soll das Textadventure auf englsich oder deutsch geschrieben werden?}
    Das Textadventure \textit{Der Zug} spielt eine zentrale Rolle innerhalb des Liverollenspiels und ist eine wichtige Ergänzung in der die Konzepte des Spiels auf einer weiteren Ebene verdeutlicht werden sollen. 
    \textit{Der Zug} zielt sowohl darauf ab, verschiedene Medien in Liverollenspiel zu integrieren, als auch verschiedene Medien zu Vermittlung von Bildungsinhalten nutzbar zu machen, bzw. die Benutzbarkeit verschiedener Medien zur Vermittlung sicht- und denkbar zu machen.\\
    Stickpunkte (Wozu ist das Textadventure da):
    \begin{itemize}
    \item Darstellung des Unterbewusstseins (Traum)
    \item Beeinflussung der Spieler*innen durch das textadventure
    \item Einleitung in das Spiel (ersten Akt)
    \item (Geheime) Begegnung mit anderen Spieler*innen (weitere Akte)
    \item Zusätzliche Informationen für die Spieler*innen
    \item Entwicklung des Bewusstseins (Aspekte dialektischer Philosophie einbinden)
    \end{itemize}
    Negative Aspekte:
    \begin{itemize}
    \item Kampfsystem?
    \item Nimmt es in Zukunft zu viel Raum ein? 
    \end{itemize}
    Während das Liverollenspiel selbst äußerlich in zwei Teile aufgeteilt ist (\textit{Der Zug} und \textit{Die Anstalt}), verdeutlicht das eingegliederte Textadventure die innere Aufteilung in drei Akte: 
    \begin{itemize}
    \item[I] Der Raum
    \item[II] Das Kollektiv 
    \item[III] Die Gesellschaft
    \end{itemize}
    Während in dem Liverollenspiel selbst diese drei Akte sich nacheinander entfalten, greift das Spiel auf jeder Ebene 1. in abstrakter Form/ auf einer Metaebene auf die nächste Ebene zu und öffnet 2. währende der Öffnung schon hin zum nächsten Akt. \\
    Im Gegensatz zu \glqq gewöhnlichen\grqq{} Textadventure, ist es eher angelehnt an ein Openworld Rollenspiel, in dem sich der Charakter zu jeder Zeit frei bewegen kann.
    Das heißt, er befindet sich in einem Raum und kann sich mittels verschiedene \textit{Befehle} z.B. Gegenstände oder Personen in seinem Raum anzeigen lassen, er kann mit diesen in Interaktion treten/ sie ins Inventar aufnehmen, er kann sich die benachbarten Räume anzeigen lassen und diese Wechseln. 
    Beim Betreten eines Raumes, wird eine grobe Beschreibung des Raumes, dessen, was der Charakter sieht, ausgegeben. 
    Es wird also versucht die Welt eines gewöhnlichen (PC-) Rollenspiels mittels literarischer Beschreibung zu visualisieren. \\
    Der Aktwechsel innerhalb des Textadventures geschieht parallel zu dem realen Aktwechseln. \todo{Bilder einf\"ugen vom Spiel}

    \subsection{Der Raum}
    Wie die Charaktere zu Beginn in einzelnen (oder Gruppen-) Räumen sind, befindet sich der Spieler/ die Spielerin zu Beginn alleine in einem Zugabteil. 
    Alleine bis auf einen Begleiter, \textit{Parsen}.
    Der Charakter in seinem Abteil, kann dieses nicht verlassen. 
    Der Ausgang \glqq Tür auf den Gang\grqq{} wird zwar angezeigt, aber bei dem Versuchen diesen auszuwählen, tritt der Spieler wider Erwarten nicht auf den Gang des Zuges, sondern befindet sich in dem Foyer eines Krankenhauses. 
    In der Welt in der der Spieler sich nun befindet, ist eine Zombieapokalypse ausgebrochen. \todo{Es wäre zu überlegen, ob das das beste Setting ist :D}
    Egal wie der Spieler sich entscheidet, dieses kleine \hyperref[zombieapokalypse]{Mini-Spiel} endet immer damit, dass der Spieler aus dem Krankenhaus flüchtet und wieder in seinem Abteil auftaucht.
    Das geschieht so lange, bis der Spieler auch im eigentlichen Rollenspiel seine Tür geöffnet hat.
    Ab dann betritt er bei Wählen dieses Ausgangs tatsächlich den Gang des Zuges.\\
    Wie also im Liverollenspiel selbst, bleibt der \glqq natürtliche\grqq{} Übergang in den zweiten Akt an bestimmte Ereignisse in dem \glqq realen\grqq{} ersten Akt verknüpft. 
    Allerdings eröffnet der Dialog mit dem geheimnisvollen Begleiter, Parsen, eine Möglichkeit vorab in Szenen des zweiten Akts einzutauchen: 
    \begin{itemize}
    \item[] Neben normalen Dialogoptionen, die bei jeden Charakter anders sind und die Möglichkeit bieten in die Hintergrundgeschichte des Charakters einzutauchen und neue Aspekte des Charakters kennenzulernen (also wiederrum Darstellung des Innenlebens des Charakters, wie im realen ersten Akt die Bilder, die Musik, etc. des Raumes), kann jeden Charakter im Dialog mit Parsen von einem Traum erzählen. 
    Dieser Traum entwirft ein weiteres kleines (Rollen-) Spiel, welches durch Fragen von Parsen konstruiert wird, á la \textit{und was ist dann passiert? ...}. 
    Innerhalb dieses Fragen-Antwort-Spiels, kann der jeweilige Charakter alle Charaktere seines (späteren) Kollektives treffen.
    Sofern diese gerade auch im Spiel sind, sogar real mit Ihnen schreiben. (\textit{\frqq hast du mit ihm gesprochen?\flqq{} \frqq Ja\flqq.} $\rightarrow$ Dann beginnt der Dialog mit diesem Spieler).
    Insofern ist das Textadventure teilweise ein mprpg (Multiplayer Roleplaying-Game).
    \end{itemize}

    \subsection{Das Kollektiv}
    Im zweiten Akt steht dem Spieler der Zugang zu dem Gang des Zuges frei. 
    Er kann nun also real mit allen Charakteren aus dem zuvor erzählten Traum kommunizieren und ihnen auch begegenen, außerdem aber mit allen weiteren Charakteren, also allen Charakteren, die nicht Teil seines Kollektives sind. \\
    An dieser Stelle fehlt noch ein Konzept, um die Trennung zwischen Kollektiv und Nicht-Kollektiv-Mitglieder darzustellen.
    Möglicher Konzeptvorschlag:
    \begin{itemize}
    \item[] Alle Mitgleider des Kollektives befinden sich in einem bestimmten Abschnitt des Zuges (z.B. im Bordbistro), während alle übrigen Charaktere sich in den verschiedenen Zugabteilen befinden.
    \end{itemize}

    In diesem Akt kann durch etwas, wie ein Bug auf den nächsten Akt zugegriffen werden: 
    durch einen Raum betritt man plötzlich eine scheinbar magische Welt:
    man verlässt den Zug und ist in einem anderen Setting (saftig grüne Wiesen, Wälder, Blumen, blauer Himmel etc.).
    Hier tritt erstmals eine harmonische Gesellschaft auf. 
    Allerdings besteht diese Gesellschaft zunächst nur aus den Mitgleidern des Kollektives und zeigt noch die Fehlerhaftigkeit der Gesellschaft: es gibt noch ausgeschlossene, es gibt noch totalitäre Strukturen: die Gesellschaft \textit{wirkt} wie eine Gesellschaft, ist aber noch in der Form eines Kollektives. 

    \subsection{Die Gesellschaft}
    Die Gesellschaft findet nun nur noch in dieser traumhaften Welt der saftigen Wiesen, der Blumen und des perfekt blauen Himmels statt. 
    Die Gesellschaft ist vereint und scheinbar perfekt. 
    Der noch vorhandene Widerspruch ist dargestellt durch die Möglichkeit der Rückkehr in der Zug.\\
    Ein mögliches Konzept für das Ende des Spiels wäre folgendes: falls die Spieler*innen innerhalb des eigentlichen Liverollenspiels...
    \begin{itemize}
    \item[1.] ... in eine harmonische Gesellschaft übergehen, also die Widersprüche auflösen können, wird im Textadventure beschrieben, wie der Zug abfährt, während alle Spieler*innen auf den saftigen Wiesen und Feldern stehen. 
    \item[2.] ... den Zerfall der Gemeinschaft herbeiführen, versinkt alles im Chaos: die harmonische Welt zerfällt, die Spieler*innen befinden sich plötzlich wieder in ihrem Zugabteil, der Zug fährt los und entgleist.
    \end{itemize}

    \subsection{Mini-Spiel: Zombieapokalypse} \label{zombieapokalypse}
    Idee des Spiels:
    \begin{itemize}
    \item Verbindung zur realen Welt (Kamp gegen das System)
    \item Kamp gegen die sieben Todsünden
    \end{itemize}
    Zumindest in diesem Teil des Textadventures wird es ein Kampsystemgeben. \todo{Frage wäre, ob es ein Kampfsystem vielleicht auch ansonsten gibt}
    Der Spieler befindet sich in dem Foyer eines Krnakenhauses und entdeckt langsam, dass die meisten Menschen dort Zombies sind, die gerade dabei sind die Ärzte zu überwältigen. 
    Evtl. könnte der Spieler die Option haben sich den Ärzten oder dne Zombies anzuschließen.
    Egal welche Option er wählt, dieses kleine Mini-Spiel wird immer damit enden, dass der Spieler wieder in seinem Abteil endet, nachdem er versucht aus dem Krankenhaus zu flüchten.
    
    \subsection{Leitfaden: Spiel erstellen}
    Hier wird aufgeführt, wie Charactere, Räume etc. erstellt werdne können.
    
    \subsection{Leitenfaden: Spielhilfe}
    Hier wird aufgeführt, welche Befehle es gibt und wie \textit{der Spieler} das spiel benutzen kann.
    

    \section{OT-Mechanism}

    \subsection{Cut} \label{cut}
    Der Cut ist ein OT-Mechanismus, der das Spiel für eine gewisse Zeit unterbricht und (optional) die Möglichkeit bietet an dieser Stelle das Spiel mit allen Teilnehmer*innen OT zu reflektieren. 

    \section{Charaktere}
    Hier werden alle Charaktere ausführlich aufgeführt. Die Charaktere sind Hauptbestandteil des Spieles. Wichtig ist noch zu entscheiden, welches Aspekte jedes Charakters \glqq real\grqq{} sind und welche für das jeweilige Experiment \glqq eingeimpft\grqq{} wurden.\\
    Ideen für Charaktere:
    \begin{itemize}
    \item Denkt er ist durchführer
    \item Denkt er ist Patient, aber findet heraus, dass er Arzt ist
    \item Balletschule (Keine Ahnung mehr, was das war. Vielleicht jemand, der denkt, er/sie leitet eine Balletschule?)
    \item Ein Charaktere, der Stumm ist.
    \end{itemize}

    \subsection{Johnny-Maria Murdo-Gonzales}
    Völlig durchgeknallter Kameraman, hört nie auf alles zu Filmen, rennt rum, schneidet Grimassen, lässt die anderen Grimassen schneiden, völlig irre der Typ, durchgeknallt! 
    Er ist allerdings der oberste Leiter des Experiments \glqq spielt\grqq{} den verrückten Kameramann. 
    
    \subsection{Victoria Sackville}
    Schriftstellerin, die lieber im 19. Jahrhundert gelebt hätte. Sie liebt Hunde; Menschen sind Victoria eigentlich zu kompliziert, unzuverlässige Gestalten, die einen immer irgendwann verlassen. Ihre engste Freundin nennt sie Vita, aufgrund ihres lebhaften Gemüts und der Angewohnheit sich in alle Situationen vollständig zu stürzen.

    \section{Lerninhalte}
    Betrachtet man die aktuelle gesellschaftliche und politische Situation stellt sich bei einem  schnell ein Gefühl der  Ohnmacht bzw. der Machtlosigkeit ein: nicht nur in Deutschland, sondern global gesehen erstarken rechtspopulistische Kräfte, Meinungen und Parteien, nicht nur bahnt sich eine Klimakatastrophe an, diese wird auch noch von führenden Politiker*innen auf der ganzen Welt nicht nur nicht ernst genommen, sondern teilweise bewusst ignoriert oder geleugnet, die neoliberale Wirtschaft führt zur fortwährenden Konzentration des Kapitals in wenigen Händen, während der andere Teil der Gesellschaft mit Niedriglöhnen, Wohnungsnot und Arbeitslosigkeit zu kämpfen hat; gleichzeitig werden allerorts bahnbrechende neue Technologie entwickelt, die bei der Bekämpfung dieser Probleme helfen könnten. 
    Davon ist allerdings für die Ärmsten der Gesellschaft nichts zu spüren. Stattdessen wird über Aussichtslosigkeit, Flüchtlinge und die Eliten/ das Establishment geschimpft. Gerade, dass eine Kritik der aktuellen wirtschaftlichen Verhältnisse, auch aus der Linken, häufig antisemitisch geprägt ist, oder von Verschwörungstheoretikern geführt wird und ansonsten die gegebenen Verhältnisse als unantastbar und notwendig dargestellt werden, wirft die Frage auf, warum ein grundsätzliches Umdenken einer Wirtschaftsform, die fortwährend Rechtsextremismus, Klimakatastrophen und Ungleichheit produziert bei einem großen Teil der Menschen ausbleibt. \\
    Wir wollen in \textit{Bebarengan Sepur – Entlang der Schienen} eine Erklärung liefern für die scheinbare Notwendigkeit des Bestehenden, sowie in der dazu führenden grundlegenden Tendenz den Nährboden für rechtes, absolutistisches und menschenfeindliches Denken nachweisen. 
    Dabei beziehen wir uns maßgeblich auf die Tradition dialektischer Philosophie von Hegel über Marx und Nietzsche zu Adorno. 
    Das Vermittelte teilen wir in drei Kategorien: zunächst soll ein Grundverständnis für einige zentrale Aspekte der dialektischen Philosophie geschaffen werden, danach soll unter den Titel \textit{Schein und Zweite Natur}, der Schein der Notwendigkeit des Bestehenden erklärt und durchbrochen werden. 
    In dem letzten Abschnitt sollen diese Strukturen in einen Zusammenhang mit dem Thema Rechtsextremismus gebracht werden.

    \subsection{Einführung in dialektisches Denken: Das Nichtidentische}
    Eine grundlegenden Einführung in dialektisches Denken wollen wir über den Begriff der Nichtidentität Adornos liefern. 
    Dazu bedienen wir uns der sechs Dimensionen die Jürgen Ritsert dem Begriff der Nichtidentität zuordnet. 
    Diese wollen wir in kürze zusammenfassend darlegen. \\
    Nichtidentität bezeichnet auf der sogenannten ontologischen Ebene (auf der Ebene des Seins) das dem Denken oder der Sprache Entgegengesetzte: Materie. 
    Diese muss im Akt der Erkenntnis für den Menschen in eine andere Sphäre übersetzt werden (wir haben schließlich keinen Baum im Kopf, sondern bloß den Begriff eines Baumes). 
    In der erkenntnistheoretischen Dimension wird die Frage behandelt, wie wir (Erkenntnisinstanz, Subjekt) einen Gegenstand (Erkenntnisgegenstand, Objekt) überhaupt erkennen können. 
    Dialektik steht hier im Gegensatz zu dem Realismus, der einen unmittelbaren Zugang zu dem zu erkennenden Gegenstand annimmt, sowie zu Idealismus, der – in seiner strengsten Form – die Existenz von etwas Seiendem außerhalb des Denkens leugnet und zu dem erkenntnistheoretischen Relativismus, welches annimmt alles sein konstruiert. 
    Dialektik setzt allerdings etwas echt Seiendes voraus, welches aber notwendig verschieden ist, von unseren Begriffen, mit denen wir meinen es erfassen zu können. 
    Die sprachtheoretischen Dimension behandelt eben dieses Notwendig verschieden sein. 
    Das Wesen des Begriffes beschreibt Nietzsche als \glqq auf lauter ungleicher Fälle passen [zu müssen]. Jeder Begriff entsteht durch Gleichsetzen des Nicht-Gleichen\grqq. 
    Daraus folgt, dass der Begriff immer eine Beschneidung des Gegenstandes bedeutet. Der Gegenstand ist immer mehr als sein Begriff, umgekehrt, der Begriff aber auch mehr als sein Gegenstand. 
    Die gesellschaftstheoretische Dimension beschreibt die Vermittlung des Ganzen der Gesellschaft in jedem Einzelnen und die gesellschaftlichen Zwänge, die Verdinglichung und die Repressionen, welche die Lebensbedingungen der Individuen durch diese Vermittlung beeinflussen. 
    Das Nichtidentische steht für dieses Objektive in der gesellschaftlichen Realität, andererseits aber auch \glqq als Chiffre für Möglichkeiten der Befreiung aus strukturellem Zwang und Repression in einer Gesellschaft\grqq. 
    Die 5. Dimension, die ideologiekritische, beschreibt die Kritik an starrem, unreflektiertem Denken, also die Kritik an verdinglichtem Bewusstsein. 
    Daraus mündet für Adorno allerdings auch die Frage nach der vernünftigen Einrichtung der Welt und der Realisierung der Forderungen der Aufklärung: Freiheit, Gerechtigkeit und Gleichheit. 
    Die letzte von Ritsert aufgeführte Dimension ist die dialektische Dimension. Hier ist die Verbindungslinie zwischen dem Begriff der Nichtidentität und dem Programm Adornos negativer Dialektik gelegt. 
    Negative Dialektik bedeutet nicht nur die Hinwendung zum Nichtidentischen, sondern ebenso die Notwendigkeit dies innerhalb der Identitätsphilosophie, also mit Begriffen zu leisten. 
    Daraus ergibt sein ein schwieriges Vorhaben, \glqq nämlich gerade das, was sich nicht unmittelbar konstruieren lässt, gleichwohl zu konstruieren […] und durch die Ratio selber sich über den Gegensatz des Rationalen und Irrationalen zu Erheben\grqq.

    \subsection{Schein und Zweite Natur} \label{zweite-natur}
    Nach Menke bedeutet zweite Natur zum einem \textit{zweite} Natur, in dem Sinne, dass die geistigen Fähigkeiten des Menscen ein kulturelles Produkt und Ergebnis einer Entwicklung des Geistes sind, andererseits aber auch zweite \textit{Natur}: denn die sozialen Praktiken/ das Sittliche und machen das \textit{Wesen} der Subjekte aus, bzw. durch den Prozess der Aneignung der Gesetze der sozialen Praktiken besitzen Subjekte erst wirklich ihr Wesen.
    Außerdem erscheinen sie als notwenig und unantastbar. 
    Sie sind also Natur, weil sie 1. das Wesen des Menschen ausmachen, und 2. weil sie, wie Natur wirken.
    Zweite Natur, oder Sittlichkeit ist im dem Sinne Aneignung des Wesen des Menschen und Befreiung des Menschen aus der ersten Natur, aber zugleich Verkehrung des Geistes in Natur innerhalb des Geistes (Der Tod des Geistes durch den Geist und in dem Geist).
    Darum liegt Freiheit zuletzt nur in der Befreiung. Und Befreiung ist das Durchstoßen dieses Scheins der Notwendigkeit.

    \subsection{Rechtsextremismus}
    Genau hier liegt auch der Anknüpfungspunkt zum Rechtsextremismus. 
    Wird das Bestehende nur affimiert, der Schein der Notwendigkeit unreflektiert angenommen, dann entstehen totalitäre und autoritäre Strukturen. 


    \section{Theoretische Bezüge}
    \subsection{Nordic Larp} \label{nordic-larp}
    \subsection{Vermittlung von Bildungsinhalten}
    \subsection{Kunst und Abstraktion} \label{kunst-abstraktion}
    \subsection{Genealogische Kritik} \label{genealogische-kritik}

    \section{Java-Reihe} \label{java-reihe}
    Die \textit{Java-Reihe} wurde 2016 mit \textit{Konkalikong - Antisitismus in Verschwörungstheorien} begonnen und mit der Entwicklung von \textit{Gawasi Gukum - Überwachen und Strafen} fortgeführt (Gawasi Gukum wurde 2018 und 2019 in Neu-Anspach durchgeführt).
    Bebarengan Sepur soll der dritte und vorerst letzte Teil der Reihe sein.\\ 
    Alle Spiele wurden bisher von dem BDP (Bund Deutscher Pfadfinder*innen) ausgetragen. 
    Die Spiele sind alle ebenfalls Teil des \textit{Weltenspieler im BDP} Projekts, welches Edu-Larps (educational Live-Action-Role-Playing) für Kinder und Jugendliche entwickelt und durchführt. \\
    Alle Spiele der Reihe experimentierten mit dem Konzept der Abstraktion zur Darstellung von Bildungsinhalten aber auch zur Erzeugung einer Spielwelt. 
    Wirklich präsent wurden diese Inhalte in \textit{Gawasi Gukum}, mit der Entwicklung von \textit{Seni}, einem abstraktem Gefängnis in dem ein alternatives Rollenspiel stattfindet, und des \textit{Spiegelsaals}.
    Außerdem griffen wir bei der Entwicklung des Spielkontepts auf Elemente genealogischer Kritiken (Z.B. Foucaults \textit{Überwachen und Strafen} oder Nietzesches \textit{Genealogie der Moral}) auf und entdeckten diese als interessante Methode zur Darstellung von Bildungsinhalten.
    In \textit{Bebarengan Sepur} sollen beide Aspekte expliziter Hauptbestandteil des Game-Designs darstellen. 

    \section{Literaturverzeichnis}

 
\bibliographystyle{plain}
\bibliography{literatur}  
\end{document}

    

