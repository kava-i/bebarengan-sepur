\documentclass[a4paper, 12pt]{scrartcl}

\usepackage[left=2.5cm,top=3cm, bottom=3cm, right=2cm]{geometry}

\usepackage{secdot} % Dots in Section Numbers
\usepackage[utf8]{inputenc}
\usepackage[T1]{fontenc}
\usepackage[ngerman]{babel}
\usepackage[european]{circuitikz}
\usepackage{graphicx} % to include Pictures
\usepackage{float}

\usepackage{fancyhdr}

\usepackage{todonotes}
\usepackage{hyperref}

\pagestyle{fancy}
\fancyhf{}
\lhead{kawa-i}
\chead{Bebarengan Sepur: Entlang der Schienen}
\rhead{\thepage}
%\cfoot{\thepage}


\definecolor{nicegreen}{rgb}{0.09, 0.45, 0.27}

\begin{document}
    \begin{titlepage}
        \begin{center}
            \large {Bebarengan Sepur}
        \end{center}
    \end{titlepage}
    \tableofcontents

    \section{Einleitung}
    \textit{Bebarengan Sepur - Entlang der Schienen} ist ein philosophisches Bildungsliverollenspiel, welches Versucht, die Themen Freiheit, Autonomie und Befreiung in einen Kontext zu bringen mit den Problemen des 21. Jahrhundert (Klimakatastrophen, Wirtschaftskrisen, Rechtsradikalismus etc.). 
    Wir verbinden charakterbasiertes Liverollenspiel, anglehnt an \hyperref[nordic-larp]{Nordic Larp}, mit Kunst, Abstraktionen und surrealen Elementene. 
    Die Erzählstruktur des Liverollenspiels greift Element genealogischer Kritik auf. 
    Damit bilden \hyperref[kunst-abstraktion]{Kunst und Abstraktion}, sowie \hyperref[genealogische-kritik]{genealogische Kritik} die beiden Hauptmethoden zur Darstellung der Bildungsinhalte. 
    Diese zwei Methoden haben wir durch die \hyperref[java-reihe]{Java-Reihe} hinweg entwickelt und weiterentwickelt und sollen in \textit{Bebarangan Sepur} das erste Mal zur vollen und spieltragenden Entfaltung kommen. 
    So dienen sie hier nicht ausschließlich zu Darstellung der Bildungsinhalte, sondern bilden einen Hauptbestandteil bei der Kontruktion des Plots und der Spielgeschichte, sowie bei Erschaffung der Spielwelt.\\
    Die Themen Freiheit, Autonomie und Befreiung greifen wir anhand der Perspektiven neuerer und älterer Kritischer Theorie (Adorno, Menke, Saar), sowie ihrer Vordenker*innen (Hegel, Marx, Nietzsche) auf. 
    Theoretischer Hintergrund ist die Notwendigkeit der Befreiung aus der zweiten Natur, zum Einen als einzigem wirklichen Moment der Freiheit (Freiheit als negation konkreter Unfreiheit), zum Anderen um fundierte Kritik an aktuellen Problemen zu formulieren.

    \section{Game-Design}
    \subsection{Spielsetting}
    Das Spiel spielt in einer (dystopischen) Zukunft in der Menschen, quasi wie ein Organspende-Ausweis, sich in dem Fall eines verorndeten Gefängnisaufenthalts, als verfügbar für Experimente (oder auch "Feldstudien") erklären können. 
    Das Spiel besteht aus zwei einzelnen \glqq Akten\grqq, die jeweils zwei verschiedene solcher Experimente darstellen.
    \subsection{Die Kollektive}
    Im Laufe des Spiels bilden die Spieler*innen anhand ihrer Charakterbeschreibungen, also vorherigen Zuordnung, mehrere (vermutlich drei) Kollektive. 
    Innerhalb dieser Kollektive bildet sich ein jeweiliger Kollektiv-Konsens (wie genau das passiert, ist noch unklar), man könnte auch davon sprechen, dass sich in jedem Kollektiv eine \glqq zweite Natur\grqq{} bildet. 
    Das Eingehen jedes einzelnen Charakters in das Kollektiv, soll sich auch im Charakter widerspiegel. Beispielsweise in seinem Raum. Die Identität des Kollektivs speist sich aus der der Individuen, gleichzeitig verändert sie aber auch die Individuen selbst. 
    Hier wird versucht ein Aspekt Hegels-Philosophie zu verdeutlichen. Bezug genommen wird speziell auf zwei Passagen der Hegelschen Rechtsphilosophie: \\
    \glqq Das objektiv Sittliche das an die Stelle des Abstrakten Guten tritt, ist durch die Subjektivität [...] konkrete Substanz.\grqq{} (S.161) \\
    Und:\\
    \glqq Andererseits sind sie [die Gesetze des des objektiv Sittlichen] dem Subjekte nicht ein Fremdes, sondern es gibt das Zeugnis des Geistes von ihnen als von seinem eigenen Wesen.\grqq{} (S. 162)

    \subsection{Die Öffnung/ Der Cut}
    Erreichen die Spieler*inner innerhalb des zwieten Aktes des ersten Teils einen bestimmten Punkt, öffnen sich erstmals die beiden Abteile/ Etagen. 
    Es beginnt ein Countdown in Sekunden, begleitet von \textit{koyaanisqatsi} von Philipp Glas. 
    Während dieser Phase können sich erstmals alle Charaktere begegnen. 
    Am Ende des Countdowns wird ein \hyperref[cut]{Cut} initiert, das spiel geht kurz OT und alle Spieler*innen werden wieder auf ihr Zimmer geschickt. 
    Evtl. gehen alle Spieler*innen schlafen.  

    \section{Part One - Der Zug}
    \subsection{Spielsetting}
    \subsection{Spielrealität}
    \subsection{Metaebene}
    \subsection{Mechanismen}

    \section{Part Two - Die Anstalt}
    \subsection{Spielsetting}
    \subsection{Spielrealität}
    \subsection{Metaebene}
    \subsection{Mechanismen}

    \section{Textadventure}
    Das Textadventure \textit{Der Zug} spielt eine zentrale Rolle innerhalb des Liverollenspiels und ist eine wichtige Ergänzung in der die Konzepte des Spiels auf einer weiteren Ebene verdeutlicht werden sollen. 
    \textit{Der Zug} zielt sowohl darauf ab, verschiedene Medien in Liverollenspiel zu entegrieren, als auch verschiedene Medien zu Vermittlung von Bildungsinhalten nutzbar zu machen, bzw. die Benutzbarkeit verschiedener Medien zur Vermittlung sicht- und denkbar zu machen.\\
    Während das Liverollenspiel selbst äußerlich in zwei Teile aufgeteilt ist (\textit{Der Zug} und \textit{Die Anstalt}), verdeutlicht das eingegliederte Textadventure die innere Aufteilung in drei Akte: 
    \begin{itemize}
    \item[I] Der Raum
    \item[II] Das Kollektiv 
    \item[III] Die Gesellschaft
    \end{itemize}
    Während in dem Liverollenspiel selbst diese drei Akte sich nacheinander entfalten, greift das Spiel auf jeder Ebene 1. in abstrakter Form/ auf einer Metaebene auf die nächste Ebene zu und öffnet 2. währende der Öffnung schon hin zum nächsten Akt. 
    Im Gegensatz zu \glqq gewöhnlichen\grqq{} Textadventure, ist es eher angelehnt an ein Openworld Rollenspiel, in dem sich der Charakter zu jeder Zeit frei bewegen kann.
    Das heißt, er befindet sich in einem Raum und kann sich mittels verschiedene \textit{Befehle}, z.B. Gegenstände oder Personen in seinem Raum anzeigen lassen, er kann mit diesen in Interaktion treten/ sie ins Inventar aufnehmen, er kann sich die benachbarten Räume anzeigen lassen und diese Wechseln. 
    Beim Betreten eines Raumes, wir eine grobe Beschreibung des Raumes, dessen, was der Charakter sieht, ausgegeben. 
    Es wird also versucht die Welt eines gewöhnlichen (PC-) Rollenspiels mittels litararischer Beschreibung zu imaginieren. \\
    Der Aktwechsel innerhalb des Textadventures geschieht parallel zu dem realen Aktwechseln.

    \subsection{Der Raum}
    Wie die Charaktere zu Begin in einzelnen (oder Gruppen-) Räumen sind, befindet sich der Spieler/ die Spielerin zu Begin alleine in einem Zugabteil. 
    Alleine bis auf einen Begleiter, \textit{Parsen}.
    Der Charakter in seinem Abteil, kann dieses nicht verlassen. 
    Der Ausgang \glqq Tür auf den Gang\grqq{} wird zwar angezeigt, aber bei dem Versuchen diesen Auszuwählen, wir erfolgt die Ausgabe: \textit{Diese Tür ist verschlossen, finde den passenden Schlüssel in die, um sie zu öffnen}.
    Wie also im Liverollenspiel selbst, bleibt der \glqq natürtliche\grqq{} Übergang in den zweiten Akt an bestimmte Ereignisse in dem \glqq realen\grqq{} ersten Akt verknüpft. 
    Allerdings eröffnet der Dialog mit dem geheimnisvollen Begleiter, Parsen, eine Möglichkeit vorab in Szenen des zweiten Akts einzutauchen: \\
    Neben normalen Dialogoptionen, die bei jeden Charakter anders sind und die Möglichkeit bieten in die Hintergrundgeschichte des Charakters einzutauchen und neue Aspekte des Charakters kennenzulernen (also wiederrum Darstellung des Innenlebens des Charakters, wie im realen ersten Akt die Bilder, die Musik, etc. des Raumes), kann jeden Charakter im Dialog mit Parsen von einem Traum erzählen. 
    Dieser Traum entwirft ein weiteres kleines (Rollen-) Spiel, welches durch Fragen von Parsen konstruiert wird, á la \textit{und was ist dann passiert? ...}. 
    Innerhalb dieses Fragen-Antwort-Spiels, kann der jeweilige Charakter alle Charaktere seines (späteren) Kollektives treffen.
    Sofern diese gerade auch im Spiel sind, sogar real mit Ihnen schreiben. (\textit{\frqq hast du mit ihm gesprochen?\flqq{} \frqq Ja\flqq.} $\rightarrow$ Dann beginnt der Dialog mit diesem Spieler).
    Insofern ist das Textadventure teilweise ein mprpg (Multiplayer Roleplaying-Game).

    \subsection{Das Kollektiv}
    Im zweiten Akt steht dem Spieler der Zugang zu den Gang des Zuges frei. 
    Er kann nun also real mit allen Charakteren aus dem zuvor erzählten Traum kommunizieren und ihnen auch begegenen, außerdem aber mit allen weiteren Charakteren, also allen Charakteren, die nicht Teil seines Kollektives sind. \\
    An dieser Stelle fehlt noch ein Konzept, um die Trennung zwischen Kollektiv und Nicht-Kollektiv-Mitglieder darzustellen.
    MöglicheKOnzeptvorschlag:
        \begin{itemize}
        \item[] Alle Mitgleider des Kollektives befinden sich in einem bestimmten Abschnitt des Zuges (z.B. im Bordbistro), während alle übrigen Charaktere sich in den verschiedenen Zugabteilen befinden.
        \end{itemize}
    In diesem Akt kann durch etwas, wie ein Bug auf den nächsten Akt zugegriffen werden: 
    durch einen Raum betritt man plötzlich eine scheinbar magische Welt:
    man verlässt den Zug und ist in einem anderen Setting (saftig grüne Wiesen, Wälder, Blumen, blauer Himmel etc.).
    Hier tritt erstmals eine harmonische Gesellschaft auf. 
    Allerdings besteht diese Gesellschaft zunächst nur aus den Mitgleidern des Kollektives und zeigt noch die Fehlerhaftigkeit der Gesellschaft: es gibt noch ausgeschlossene, es gibt noch totalitäre Strukturen: die Gesellschaft \textit{wirkt} wie eine Gesellschaft, ist aber noch in der Form eines Kollektives. 

    \subsection{Die Gesellschaft}
    Die Gesellschaft findet nun nur noch in dieser traumhaften Welt der saftigen Wiesen, der Blumen und des perfekt blauen Himmels statt. 
    Die Gesellschaft ist vereint und scheinbar perfekt. 
    Der noch vorhandene Widerspruch ist dargestellt durch die Möglichkeit der Rückkehr in der Zug.

    \section{OT-Mechanism}
    \subsection{Cut} \label{cut}
    Der Cut ist ein OT-Mechanismus, der das Spiel für eine gewisse Zeit unterbricht und (optinal) die Möflichkeit bietet an dieser Stelle das Spiel mit allen Teilnehmer*innen OT zu reflektieren. 

    \section{Charaktere}

    \section{Lerninhalte}

    \section{Theoretische Bezüge}
    \subsection{Nordic Larp} \label{nordic-larp}
    \subsection{Vermittlung von Bildungsinhalten}
    \subsection{Kunst und Abstraktion} \label{kunst-abstraktion}
    \subsection{Genealogische Kritik} \label{genealogische-kritik}

    \section{Java-Reihe} \label{java-reihe}
    Die \textit{Java-Reihe} wurde 2016 mit \textit{Konkalikong - Antisitismus in Verschwörungstheorien} begonnen und mit der Entwicklung von \textit{Gawasi Gukum - Überwachen und Strafen} fortgeführt (Gawasi Gukum wurde 2018 und 2019 in Neu-Anspach durchgeführt).
    Bebarengan Sepur soll der dritte und vorerst letzte Teil der Reihe sein.\\ 
    Alle Spiele wurden bisher von dem BDP (Bund Deutscher Pfadfinder*innen) ausgetragen. 
    Die Spiele sind alle ebenfalls Teil des \textit{Weltenspieler im BDP} Projekts, welches Edu-Larps (educational Live-Action-Role-Playing) für Kinder und Jugendliche entwickelt und durchführt. \\
    Alle Spiele der Reihe experimentierten mit dem Konzept der Abstraktion zur Darstellung von Bildungsinhalten aber auch zur Erzeugung einer Spielwelt. 
    Wirklich präsent wurden diese Inhalte in \textit{Gawasi Gukum}, mit der Entwicklung von \textit{Seni}, einem abstraktem Gefängnis in dem ein alternatives Rollenspiel stattfindet, und des \textit{Spiegelsaals}.
    Außerdem griffen wir bei der Entwicklung des Spielkontepts auf Elemente genealogischer Kritiken (Z.B. Foucaults \textit{Überwachen und Strafen} oder Nietzesches \textit{Genealogie der Moral}) auf und entdeckten diese als interessante Methode zur Darstellung von Bildungsinhalten.
    In \textit{Bebarengan Sepur} sollen beide Aspekte expliziter Hauptbestandteil des Game-Designs darstellen. 

    \section{Literaturverzeichnis}

 
\end{document}
