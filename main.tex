\documentclass[a4paper, 12pt]{scrartcl}

\usepackage[left=2.5cm,top=3cm, bottom=3cm, right=2cm]{geometry}

\usepackage{secdot} % Dots in Section Numbers
\usepackage[utf8]{inputenc}
\usepackage[T1]{fontenc}
\usepackage[ngerman]{babel}
\usepackage[european]{circuitikz}
\usepackage{graphicx} % to include Pictures
\usepackage{float}

\usepackage{fancyhdr}

\usepackage{todonotes}
\usepackage{hyperref}

\pagestyle{fancy}
\fancyhf{}
\lhead{kawa-i}
\chead{Bebarengan Sepur: Entlang der Schienen}
\rhead{\thepage}
%\cfoot{\thepage}


\definecolor{nicegreen}{rgb}{0.09, 0.45, 0.27}

\begin{document}
    \begin{titlepage}
        \begin{center}
            \large {Bebarengan Sepur}
        \end{center}
    \end{titlepage}
    \tableofcontents

    \section{Einleitung}
    \textit{Bebarengan Sepur - Entlang der Schienen} ist ein philosophisches Bildungsliverollenspiel, welches Versucht, die Themen Freiheit, Autonomie und Befreiung in einen Kontext zu bringen mit den Problemen des 21. Jahrhundert (Klimakatastrophen, Wirtschaftskrisen, Rechtsradikalismus etc.). 
    Wir verbinden charakterbasiertes LARP, anglehnt an \hyperref[nordic-larp]{Nordic Larp}, mit Kunst, Abstraktionen und surrealen Elementene. 
    Die Erzählstruktur des Liverollenspiels greift Element genealogischer Kritik auf. 
    Damit bilden \hyperref[kunst-abstraktion]{Kunst und Abstraktion}, sowie \hyperref[genealogische-kritik]{genealogische Kritik} die beiden Hauptmethoden zur Darstellung der Bildungsinhalte. 
    Diese zwei Methoden haben wir durch die \hyperref[java-reihe]{Java-Reihe} hinweg entwickelt und weiterentwickelt und sollen in \textit{Bebarangan Sepur} das erste Mal zur vollen und spieltragenden Entfaltung kommen. 
    Die Themen Freiheit, Autonomie und Befreiung greifen wir anhand der Perspektiven neuerer und älterer Kritischer Theorie, sowie ihrer Vordenker*innen (Hegel, Marx, Nietzsche) auf. 
    Theoretischer Hintergrund ist die Notwendigkeit der Befreiung aus der zweiten Natur, zum Einen als einzigem wirklichen Moment der Freiheit (Freiheit als negation konkreter Unfreiheit), zum Anderen um fundierte Kritik an aktuellen Problemen zu formulieren.

    \section{Game-Design}
    \subsection{Spielsetting}
    Das Spiel spielt in einer (dystopischen) Zukunft in der Menschen, quasi wie ein Organspende-Ausweis, sich in dem Fall eines verorndeten Gefängnisaufenthalts, als verfügbar für Experimente (oder auch "Feldstudien") erklären können. Das Spiel besteht aus zwei einzelnen "Akten", die jeweils zwei verschiedene solcher Experimente darstellen.
    \subsection{Die Gruppen}

    \section{Part One - Der Zug}
    \subsection{Spielsetting}
    \subsection{Spielrealität}
    \subsection{Metaebene}
    \subsection{Mechanismen}

    \section{Part Two - Die Anstalt}
    \subsection{Spielsetting}
    \subsection{Spielrealität}
    \subsection{Metaebene}
    \subsection{Mechanismen}


    \section{Charaktere}

    \section{Lerninhalte}

    \section{Theoretische Bezüge}
    \subsection{Nordic Larp} \label{nordic-larp}
    \subsection{Vermittlung von Bildungsinhalten}
    \subsection{Kunst und Abstraktion} \label{kunst-abstraktion}
    \subsection{Genealogische Kritik} \label{genealogische-kritik}

    \section{Java-Reihe} \label{java-reihe}
    Die \textit{Java-Reihe} wurde 2016 mit \textit{Konkalikong - Antisitismus in Verschwörungstheorien} begonnen und mit der Entwicklung von \textit{Gawasi Gukum - Überwachen und Strafen} fortgeführt (Gawasi Gukum wurde 2018 und 2019 in Neu-Anspach durchgeführt).
    Bebarengan Sepur soll der dritte und vorerst letzte Teil der Reihe sein.\\ 
    Alle Spiele wurden bisher von dem BDP (Bund Deutscher Pfadfinder*innen) ausgetragen. 
    Die Spiele sind alle ebenfalls Teil des Weltenspieler im BDP Projekts, welches Edu-Larps (educational Live-Action-Role-Playing) für Kinder und Jugendliche entwickelt und durchführt. \\
    Alle Reihe experimentierten mit dem Konzept der Abstraktion zu Darstellung von Bildungsinhalten aber auch zur Erzeugung einer Spielwelt. 
    Wirklich präsent wurden diese Inhalte in \textit{Gawasi Gukum}, mit der Entwicklung von \textit{Seni}, einem abstraktem Gefängnis in dem ein alternative Rollenspiel stattfindet und des Spiegelsaals.
    Außerdem griffen wir bei der Entwicklung des Spielkontepts auf Elemente genealogischer Kritiken (Z.B. Foucaults \textit{Überwachen und Strafen} oder Nietzesches \textit{Genealogie der Moral}) auf und entdeckten diese als Interessante Methode zur Darstellung von Bildungsinhalten.
    In \textit{Bebarengan Sepur} sollen beide Aspekte expliziter Hauptbestandteil des Game-Designs darstellen. 


    
    \section{To Do}
    \subsection{Spielsetting}
    \subsection{Spielrealität}
    \subsection{Metaebene}
    \subsection{Mechanismen}
    \subsection{Charaktere}
 
\end{document}
